\documentclass[11pt, letterpaper]{article}
\usepackage[french]{babel}
\usepackage[a4paper]{geometry}
\usepackage{hyperref}
\usepackage{graphicx}
\usepackage{subcaption}
\usepackage{titling}

\geometry{
    top=1.78cm,
    bottom=1.78cm,
    left=1.65cm,
    right=1.65cm
}

\newcommand{\subtitle}[1]{%
    \posttitle{%
        \par\end{center}
        \begin{center}\large#1\end{center}
        \vskip0.5cm
    }%
}

\title{Modélisation des mondes virtuels : \\
    Rapport Millestone}
\author{Bomard Stéphane \\
        {\tt\small 12006500}}

\begin{document}
    \maketitle
    
    \section{Contexte}
        Pour la partie dédiée au millestone, nous nous appuierons sur la carte de hauteur affichée ci-dessous.
        Enfin, la base de code est présente \href{https://github.com/Styami/Modelisation-mondes-virtuels.git}{ici}.
        \begin{figure}[!h]
            \centering 
            \includegraphics[scale=0.1]{data/heightmap.png} 
            \caption{Carte de référence}
            \label{fig:ref}
        \end{figure}\hfill
        \\
        La carte utilisée représente du 1081 * 1081 et chacune des cartes présentées aura été normalisée entre 0 et 1.
        De plus, l'exécution du programme se fait en environs 8 secondes sur une machine équipée d'un CPU AMD Ryzen 7 7840HS.


    \section{millestone}
        Tous les résultats présentés auront subi 2 blur de traitement ou 2 smooth traitement avant les calculs puis normalisés.
        De plus, il est important de noter que les images présentes ici ont subis différents types d'affinage pour qu'elles soient visibles.
        On peut notamment citer l'image du stream area qui a subie une $\sqrt[4]{x}$ avant d'être normalisée.
        
        \begin{figure}[!h]
            \centering
            \begin{subfigure}[a]{0.3\textwidth}
                \centering
                \includegraphics[width=\textwidth]{data/laplacianBlur.png}
                \caption{Laplacian}
                \label{fig:laplacian_blur}
            \end{subfigure}
            \hfill
            \begin{subfigure}[b]{0.3\textwidth}
                \centering
                \includegraphics[width=\textwidth]{data/normGradientBlur.png}
                \caption{Norme du gradient}
                \label{fig:gradient_blur}
            \end{subfigure}
            \hfill
            \begin{subfigure}[c]{0.3\textwidth}
                \centering
                \includegraphics[width=\textwidth]{data/streamAreaBlur.png}
                \caption{Stream area (puissance 4)}
                \label{fig:stream_blur}
            \end{subfigure}
            \label{fig:blur_map}
            \caption{Carte sous blur}
        \end{figure}
        
        \begin{figure}[!h]
            \begin{subfigure}[a]{0.3\textwidth}
                \centering
                \includegraphics[width=\textwidth]{data/laplacianSmooth.png}
                \caption{Laplacian}
                \label{fig:laplacian_smooth}
            \end{subfigure}
            \hfill
            \begin{subfigure}[b]{0.3\textwidth}
                \centering
                \includegraphics[width=\textwidth]{data/normGradientSmooth.png}
                \caption{Norme du gradient}
                \label{fig:gradient_smooth}
            \end{subfigure}
            \hfill
            \begin{subfigure}[c]{0.3\textwidth}
                \centering
                \includegraphics[width=\textwidth]{data/streamAreaSmooth.png}
                \caption{Stream area (puissance 4)}
                \label{fig:stream_smooth}
            \end{subfigure}
            \label{fig:smooth_map}
            \caption{Carte sous smooth}
        \end{figure}
        
        \break
        Norme du gradient :

        Plus la couleur vire vers au noir, plus la slope est élevée.

        Stream Area :

        Plus la couleur tends vers le blanc, plus l'aire de drainage est forte.

        Laplacien :

        Lorsque la couleur du laplacien tends vers le bleu, cela implique que ce dernier est négatif et inversement pour la couleur rouge.
        Enfin, la couleur blanche implique un laplacien proche de 0. 
        
\end{document}