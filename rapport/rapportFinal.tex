\documentclass[11pt, letterpaper]{article}
\usepackage[french]{babel}
\usepackage[a4paper]{geometry}
\usepackage{hyperref}
\usepackage{graphicx}
\usepackage{subcaption}
\usepackage{titling}

\geometry{
    top=1.78cm,
    bottom=1.78cm,
    left=1.65cm,
    right=1.65cm
}

\newcommand{\subtitle}[1]{%
    \posttitle{%
        \par\end{center}
        \begin{center}\large#1\end{center}
        \vskip0.5cm
    }%
}

\title{Modélisation des mondes virtuels : \\
    Rapport final}
\author{Bomard Stéphane \\
        {\tt\small 12006500}}

\begin{document}
    \maketitle
    
    \section{Contexte}
        Pour la partie dédiée à l'érosion, nous nous appuierons sur la carte de hauteur affichée ci-dessous.
        Enfin, la base de code est présente \href{https://github.com/Styami/Modelisation-mondes-virtuels.git}{ici}.
        \begin{figure}[!h]
            \centering 
            \includegraphics[scale=0.1]{data/heightmap.png} 
            \caption{Carte de référence}
            \label{fig:ref}
        \end{figure}\hfill
        \\
        La carte utilisée représente du 1081 * 1081 et chacune des cartes présentées aura été normalisée entre 0 et 1.
        De plus, l'exécution du programme se fait en environs 2.4 secondes sur une machine équipée d'un CPU AMD Ryzen 7 7840HS.
        \\

    \section{Érosion}
        Les résultats présentés seront issus d'une carte de hauteurs ayant subi une hill slope erosion
        ainsi qu'une stream power érosion avec comme puissance de 4 pour la stream area et une puissance de 3 sur la slope.
        \pagebreak
        \begin{figure}[!h]
            \centering
            \begin{minipage}{0.35\linewidth}
                \includegraphics[width=\linewidth]{data/heightmap.png} 
                \caption{carte d'origine}
                \label{fig:posTex}
            \end{minipage}\hfill
            \begin{minipage}{0.35\linewidth}
                \includegraphics[width=\linewidth]{data/thermalErode.png}
                \caption{Large scale stream power érosion}
                \label{fig:uvTex}
            \end{minipage}
            \begin{minipage}{0.45\linewidth}
                \centering
                \includegraphics[width=\linewidth]{rendu.png}
                \caption{rendu sur le logiciel Blender}
                \label{fig:blender}
            \end{minipage}
        \end{figure}
        
        On observe que la carte de droite présente des aspérités plus cisaillantes.
        Toutefois, les rainures ne sont pas assez notables. Les pistes d'améliorations seraient de trouver
        des puissances pour la stream area ainsi que le slope plus adéquates afin que des rainures plus fortes apparaissent
        ou de trouver une autre façon de normaliser la carte afin que les résultats soient plus notifiables.
        

        

        
\end{document}